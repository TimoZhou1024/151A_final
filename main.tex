\documentclass{article}

% Language setting
% Replace `english' with e.g. `spanish' to change the document language
\usepackage[english]{babel}

% Set page size and margins
% Replace `letterpaper' with `a4paper' for UK/EU standard size
\usepackage[letterpaper,top=2cm,bottom=2cm,left=3cm,right=3cm,marginparwidth=1.75cm]{geometry}

% Useful packages
\usepackage{amsmath}
\usepackage{graphicx}
\usepackage[colorlinks=true, allcolors=blue]{hyperref}

\title{Studying the Association Between Sleep and Depression Outcomes}
\author{Tianyun Jiang, Siyuan Zhou, Ethan Hardcastle}
\date{}

\begin{document}
\maketitle

\section*{Introduction}
\begin{enumerate}
    \item \textbf{Research Question:} We aim to study the association between sleep and depression outcomes among U.S. adults using the NHANES 2017-2020 ("Pre-Pandemic") data. Specifically: How do sleep duration and sleep trouble (e.g., snoring, gasping, diagnosed sleep disorders) relate to depressive symptom severity (quantified by Patient Health Questionnaire 9 [PHQ-9] total score), after adjusting for demographics, lifestyle factors, and comorbidities? 
    
	\item \textbf{Question Relevance:} Depression is a leading cause of disability and health burden in the U.S. Sleep duration and sleep quality are potentially modifiable risk factors that are strongly correlated with mental health. Clarifying their independent associations with depression severity, while rigorously adjusting for confounding variables, can inform clinical screening, patient education (e.g., sleep hygiene), and population-level policy. Using NHANES enables nationally representative inference for noninstitutionalized U.S. adults and supports subgroup exploration by age, sex, and race/Hispanic origin.
    
	\item \textbf{Related Work:} Prior studies consistently report that poor sleep quality is associated with more serious depressive symptoms. For example, Joo et al. (2022) reported a strong link between sleep quality and depressive symptoms in adults (Journal of Affective Disorders, 310, 258--265; \href{https://doi.org/10.1016/j.jad.2022.05.004}{doi:10.1016/j.jad.2022.05.004}). But few NHANES-based studies have jointly modeled sleep quantity and quality with depression severity under full confounder control. Building on this literature, our analysis will use PHQ-9 to construct a total depression score (0--27) and explore its associations with sleep duration and sleep-trouble indicators, progressing from simple models to fully adjusted hierarchical models with rigorous diagnostics.
\end{enumerate}

\section*{Final Regression Model \& Analysis}
\subsection*{Choice of Model}
After a series of iterations and evaluation, eventually we decide to apply Negative Binomial (NB) Regression to our dataset. Negative Binomial Regression is a type of Generalized Linear Model (GLM) specifically designed for count data. Specifically, we choose this model for three main reasons:\\
(1)Count Outcome: The PHQ-9 total score is a sum of discrete items (0-27), making it a count variable rather than a truly continuous one.\\
(2)Skewness and many Zeros: Based on our EDA, PHQ-9 total scores in the general population (NHANES) are typically highly right-skewed with many zero or low scores. NB regression handles this distribution more effectively than standard linear regression.\\
(3)Statistical Validity: It addresses the overdispersion likely present in mental health data, ensuring that standard errors and p-values are not underestimated, which leads to more reliable conclusions about the associations between sleep quality and depression.\\
Note: In the NB model, we use a log-link function, meaning predictors are modeled as having multiplicative effects on the expected PHQ-9 score on the original scale.

\subsection*{Final Model Variables}
In our final model, the response is the total scores of the PHQ-9 Questionnaire, which measures a person's depression level (higher scores indicate more serious depression). The main regressors are sleep duration, whether showing symptoms of insomnia and whether showing symptoms of snoring or gasping. The demographic covariates include age, gender, race, income-to-poverty ratio and BMI. We also add other two types of explanatory variables to the model--lifestyles(including transport, intensity of work activities and intensity of recreation activities) and interventions(including whether taking medicines, number of medicines and whether seeking professional help from therapists). The final model also includes several interaction terms to capture effect modification between key variables. Specifically, we include interactions between:

\begin{itemize}
    \item insomnia status and work intensity,
    \item insomnia status and prescription medicine use,
     \item insomnia status and the number of prescription medications, and
    \item sleep duration and the number of prescription medicines.
\end{itemize}

\subsection*{Analysis of the Model Results}
\subsubsection*{Sleep-Related Variables}
In our final model, all coefficients of sleep-related variables are statistically significant. Specifically:
\begin{itemize}
    \item After full adjustment, each additional hour of sleep is associated with a change of -0.069  in the log-expected PHQ-9 score (p = 7.19*e-08).
    \item Holding all other variables fixed, individuals with diagnosed insomnia have a log-expected PHQ-9 score that is 0.741 unit higher compared to those without insomnia(p = 8.29*e-55).
    \item Showing frequent snoring/gasping is associated with a change of +0.158 in the log-expected PHQ-9 score (p = 2.48*e-06).
\end{itemize}
We also calculate the standardized coefficients to compare those variables' contributions to the model. The estimate of standardized coefficients for the sleep duration during weekdays and that during weekends are -0.01780313 and -0.05288938 respectively. This indicates that very likely sleep duration is not the main contributor to the explained variance. The estimate of standardized coefficients for snoring/gasping is 0.16832808, which suggests its relatively larger contribution to the model. Lastly the standardized coefficients for whether showing symptoms of insomnia is 0.81047959, which means that this variable is likely to contribute most.
\subsubsection*{Other Explanatory Variables}
We discover that there are also certain associations between lifestyle/intervention variables and depression. The coefficients of the work intensity, whether seeking professional mental help and whether taking medicines are all found to be statistically significant:
\begin{itemize}
    \item  With the rest variables fixed, one unit change in the intensity of work is linked with 0.2814895 increase in the log-expected PHQ-9 score.
    \item Holding all other variables fixed, taking prescription medicines will bring 0.0505310 decrease to the log-expected PHQ-9 score.
    \item Individuals who seek professional mental help would have a log-expected PHQ-9 score that is 0.4533444 unit higher compared to those who don't.
\end{itemize}
The last result might seem surprising at first glance, but it actually makes sense--typically people who suffer from depression are much more likely to seek professional help, thus this term is linked with higher PHQ-9 scores.
\subsubsection*{Interaction Effects} 
In general, we have found four significant interaction effects, which can be stated as follows:
\begin{itemize}
    \item There is a significant negative interaction (estimate of beta is -0.201) between insomnia and the usage of prescription medicines. While insomnia is strongly associated with higher PHQ-9 scores overall, taking prescription medication significantly mitigates this impact. Specifically, among individuals with insomnia, those who take prescription medicines have a log-expected PHQ-9 score that is 0.201 units lower than those who do not. This corresponds to $\exp(-0.201)\approx 0.82$, i.e., about an 18\% lower expected PHQ-9 score among medicated individuals with insomnia compared to unmedicated individuals with insomnia, holding other variables constant.

    \item Consistent with the binary finding, we observe a significant "dose-response" interaction (estimate of beta is -0.035). For individuals suffering from insomnia, each additional prescription medication is associated with a further 0.035 decrease in the log-expected PHQ-9 score. This suggests a cumulative buffering effect: as the intensity of medical intervention (proxied by the number of prescriptions) increases, the positive association between insomnia and depressive symptoms is progressively dampened.
    \item Again, for those who show symptoms of insomnia, there is a 0.2118190-unit decrease in the log-expected PHQ-9 score when their work intensity is one unit larger.
    This is actually interesting since higher work intensity alone is associated with higher PHQ-9 scores, yet only focusing on people with insomnia the story becomes sightly different.
    \item The coefficient for the interaction term between sleep duration and number of medicines is statistically significant, yet it's very small (0.006441151) and provides little information.
\end{itemize}
\subsection*{Analysis Limitations}
\begin{itemize}
    \item Standardized coefficients are not fully representable of variables' contributions to the model, especially when there is collinearity/high correlation among some variables. 
    \item We did not blend the dataset with the sample weight, thus some conclusions derived from our model might not generalize well among the actual population.
    \item There are still a lot of potential confounders that 
    we didn't consider. For this reason, our model is still not robust enough for casual inference.
    \item Because NHANES is cross-sectional, our results should be interpreted as associations rather than causal effects, and reverse causality (e.g., depression influencing sleep) cannot be ruled out.
\end{itemize}

\section*{Research Conclusions and Insight}

\subsection*{Conclusions}
\begin{itemize}
    \item Our analysis provides strong evidence of an association between sleep-related variables and depressive symptom severity among U.S. adults. Among the sleep variables considered, sleep duration shows a comparatively weaker association with PHQ-9 scores, while indicators of sleep quality - particularly diagnosed insomnia-exhibit much stronger relationships with depression severity. Individuals reporting insomnia or frequent snoring/gasping tend to have substantially higher expected PHQ-9 scores after adjusting for demographic, lifestyle, and medical factors.

    \item The association between sleep quality and depression is not uniform across individuals but varies meaningfully with lifestyle behaviors and medical interventions. In particular, among individuals with insomnia, both prescription medication use and higher work-related physical activity are associated with lower expected PHQ-9 scores, suggesting that the relationship between insomnia and depression severity is context-dependent rather than constant across the population.
\end{itemize}

\newpage{}

\subsection*{Recommendations and Insights}
\begin{itemize}

    \item From a clinical and screening perspective, our findings suggest that measures of sleep quality—particularly insomnia symptoms—may be more informative indicators of depression severity than sleep duration alone. Incorporating targeted sleep-quality questions into routine mental health screening could help identify individuals at elevated risk for depressive symptoms.

    \item The presence of significant interaction effects highlights that associations between sleep and depression are heterogeneous across individuals. Lifestyle behaviors and medical interventions appear to modify the relationship between insomnia and depressive symptoms, suggesting that personalized or context-aware approaches may be more effective than uniform recommendations.

    \item At a population-health level, these results emphasize the importance of considering sleep disturbances within a broader behavioral and treatment context. While sleep duration remains relevant, interventions aimed at improving sleep quality or managing insomnia may offer greater potential benefit for mitigating depressive symptom burden.
\end{itemize}

\section*{Additional Work}
\subsection*{Additional Models Fitted}

\subsubsection*{A. Linear Model Attempt and Why It Failed}
At the very beginning of our analysis, we attempted to fit the data using an ordinary least squares (OLS) linear regression model. However, fundamental diagnostic checks revealed severe violations of the key assumptions required for valid inference:

\begin{itemize}
    \item \textbf{Non-normality of residuals:} The PHQ-9 total score is a discrete count variable (0-27) with a highly right-skewed distribution concentrated at low values. The QQ-plot showed systematic deviation from normality, with the residuals exhibiting heavy right tails.

    \item \textbf{Heteroscedasticity:} The residuals vs. fitted values plot displayed a clear fan-shaped pattern, indicating variance that increases with the mean—a hallmark of count data that violates the constant variance assumption of OLS.

    \item \textbf{DHARMa diagnostic validation:} Using the DHARMa package, we simulated residuals from the linear model and conducted formal tests. The dispersion test was highly significant, confirming overdispersion. The DHARMa residuals vs. fitted plot showed pronounced curvature, indicating systematic misspecification of the mean structure.
\end{itemize}

\textit{Figure 1} shows DHARMa diagnostic plots for the initial linear model attempt, confirming these violations. These diagnostic failures led us to adopt the Negative Binomial (NB) Generalized Linear Model (GLM), which is specifically designed for overdispersed count data. The NB-GLM uses a log-link function to model the relationship between predictors and the expected count, naturally accommodating the non-negative, discrete, and right-skewed nature of PHQ-9 scores.

\subsubsection*{B. Progressive Model Building: Stage-by-Stage Expansion}
We employed a systematic hierarchical modeling strategy, progressively expanding from a base model to include lifestyle and then medical intervention covariates. At each stage, we used the Likelihood Ratio Test (LRT) to formally assess whether the added complexity significantly improved model fit.

\begin{itemize}
    \item \textbf{Stage 1 (Base Model):} Included only sleep variables (average sleep duration, insomnia diagnosis, sleep apnea symptoms) and demographic controls (sex, race, age, income-to-poverty ratio, BMI). This model established the fundamental associations between sleep and depression.

    \item \textbf{Stage 2 (+ Lifestyle Covariates):} Added five lifestyle factors: vigorous/moderate work activity, active transportation, vigorous recreational activity, and secondhand smoke exposure. LRT confirmed that these additions significantly improved fit, indicating that lifestyle factors are important confounders in the sleep-depression relationship.

    \item \textbf{Stage 3 (+ Medical Interventions):} Further expanded to include medical intervention variables: prescription medication use (binary and count), and mental health professional consultation. LRT again showed significant improvement, demonstrating that treatment variables provide additional explanatory power beyond lifestyle and sleep factors.
\end{itemize}

\subsubsection*{C. Interaction Testing: Sleep $\times$ Lifestyle and Sleep $\times$ Medical}
To investigate whether the relationships between sleep variables and depression depend on lifestyle or medical context, we systematically tested all possible two-way interactions. Using significance thresholds and correcting for multiple comparisons, we identified several significant interactions, most notably:

\begin{itemize}
    \item \textit{Insomnia $\times$ Work Vigorous Activity}: Among individuals with insomnia, higher work intensity was associated with a mitigated increase in PHQ-9 scores, suggesting that physical activity may buffer some negative effects of insomnia on depression.

    \item \textit{Insomnia $\times$ Prescription Medicine}: Taking prescription medication significantly attenuated the positive association between insomnia and depression severity.

    \item \textit{Insomnia $\times$ Number of Prescriptions}: A dose-response effect where each additional medication further reduced the insomnia-related depression burden.
\end{itemize}

These interaction findings reveal that the sleep-depression association is context-dependent, varying meaningfully with modifiable factors such as physical activity and medical treatment.
\subsection*{Additional Diagnostics Attempted}

\subsubsection*{A. Model Assumptions for Negative Binomial GLM}
The Negative Binomial (NB) regression model makes the following key assumptions:

\begin{enumerate}
    \item \textbf{Count data structure:} The response variable (PHQ-9 total score) is a non-negative integer representing counts of depressive symptoms. This is satisfied by construction.

    \item \textbf{Independence of observations:} Given the cross-sectional design of NHANES 2017-2020, observations are assumed to be independent across individuals. This is reasonable for this survey design, though we acknowledge that household clustering was not explicitly modeled.

    \item \textbf{Log-linear mean structure:} The expected count is assumed to be an exponential function of the linear predictor: $E[Y|X] = \exp(X\beta)$. This is the canonical log-link function for count models.

    \item \textbf{Overdispersion accommodation:} Unlike Poisson regression (which assumes mean = variance), the NB model includes an additional dispersion parameter to handle variance that exceeds the mean. This is critical for mental health count data, which are typically overdispersed.
\end{enumerate}

\subsubsection*{B. Diagnostic Procedures and Results}

\textbf{DHARMa Scaled Residual Diagnostics:}

To rigorously assess model adequacy, we employed the DHARMa package, which simulates scaled residuals from the fitted NB-GLM and transforms them to a uniform [0,1] scale. This approach is specifically designed for GLMs and provides more reliable diagnostics than traditional Pearson or deviance residuals.

\begin{itemize}
    \item \textbf{QQ Plot:} The quantile-quantile plot showed that observed scaled residuals closely followed the expected uniform distribution diagonal. This indicates that the NB distributional assumption is appropriate for these data. Kolmogorov-Smirnov test confirmed no significant departure from uniformity.

    \item \textbf{Residuals vs. Fitted Values:} The plot of scaled residuals against predicted values showed no systematic patterns or curvature, with the smoothed red line remaining flat near the 0.5 horizontal reference. This confirms that the log-linear mean structure is correctly specified and that there is no evidence of unmodeled nonlinearity.

    \item \textbf{Dispersion Test:} The DHARMa dispersion test compared the observed variance in the simulated residuals to the expected variance under the fitted NB model, indicating no significant residual overdispersion beyond what the NB model already accounts for. This validates our choice of NB over Poisson regression.

    \item \textbf{Outlier Test:} The DHARMa outlier test identifies observations with residuals that fall outside the simulation envelope, confirming model fit quality.
\end{itemize}

\textit{Figure 1} shows the DHARMa diagnostic plots confirming model assumptions are met.

\textbf{Variance Inflation Factors (VIF):}

To assess multicollinearity among predictors, we computed generalized VIF values for all model terms. All VIF values were below 5, indicating no concerning multicollinearity that would inflate standard errors or destabilize coefficient estimates.

\subsubsection*{C. Outliers, Leverage, and Influential Points}

\textbf{Cook's Distance Analysis:}

Cook's Distance measures how much the fitted model would change if a particular observation were removed. We used the standard threshold of $D_i > 4/n$ (where $n$ is the sample size) to identify potentially influential points.

\textit{Figures 2 and 3} show Cook's Distance plots for Stage 3 and Stage 5 models with threshold lines.

\textbf{Robustness Checks:}

To assess whether our substantive conclusions depend on these influential observations, we conducted formal robustness tests by refitting the Stage 3 and Stage 5 models after removing points with Cook's $D > 4/n$.

\textit{Stage 3 Robustness Results:}
% After removing influential points, the three main sleep variable coefficients changed by less than 10\%, confirming that our inferences are robust to these observations. All p-values remained highly significant, and the direction and magnitude of effects were substantively unchanged.
After removing influential points, the three main variable coefficients changed slightly, confirming that our inferences are robust to these observations. All p-values remained highly significant, and the direction and magnitude of effects were substantively unchanged.

\textit{Stage 5 Robustness Results:}
% Removing influential points resulted in coefficient changes of less than 10\% for weekday and weekend sleep variables. The hypothesis test for equality of weekday vs. weekend sleep coefficients remained consistent in the robust model.
Removing influential points resulted in slight coefficient changes for weekday and weekend sleep variables. The hypothesis test for equality of weekday vs. weekend sleep coefficients remained consistent in the robust model.

\textbf{Conclusion:} The comprehensive diagnostic battery—including DHARMa residual checks, VIF analysis, and Cook's Distance robustness testing—provides strong evidence that the Negative Binomial model is well-specified, that assumptions are met, and that our substantive conclusions are not driven by a few influential observations.

\section*{Figures and Tables}

\begin{figure}[h]
\centering
\includegraphics[width=0.85\textwidth]{figures/dharma_diagnostics.png}
\caption{DHARMa diagnostics: QQ plot (left) and residuals vs. fitted (right) validate NB-GLM assumptions.}
\label{fig:dharma}
\end{figure}

\begin{figure}[h]
\centering
\includegraphics[width=0.85\textwidth]{figures/phq9_distribution.png}
\caption{Distribution of PHQ-9 scores and sleep hours showing right-skewed count data.}
\label{fig:PHQ-9_dist}
\end{figure}

\begin{figure}[h]
\centering
\includegraphics[width=0.85\textwidth]{figures/sleep_phq9_loess.png}
\caption{Sleep-PHQ-9 relationship (left: LOESS smooth; right: insomnia comparison).}
\label{fig:sleep_PHQ-9}
\end{figure}

\begin{figure}[h]
\centering
\includegraphics[width=0.6\textwidth]{figures/cook_distance_stage3.png}
\caption{Cook's Distance for Stage 3 (4/n threshold shown).}
\label{fig:cook_stage3}
\end{figure}

\begin{figure}[h]
\centering
\includegraphics[width=0.6\textwidth]{figures/cook_distance_stage5.png}
\caption{Cook's Distance for Stage 5 (4/n threshold shown).}
\label{fig:cook_stage5}
\end{figure}

\begin{table}[h]
\centering
\caption{Model comparison across stages}
\label{tab:model_comparison}
\small
\begin{tabular}{lcccc}
\hline
\textbf{Stage} & \textbf{Variables} & \textbf{AIC} & \textbf{BIC} & \textbf{$\Delta$AIC} \\
\hline
1 & Sleep + Demographics & 40524.21 & 40625.34 & -- \\
2 & + Lifestyle & 40462.98 & 40607.45 & -61.23 \\
3 & + Medical & 40271.04 & 40444.41 & -191.94 \\
\hline
\end{tabular}
\end{table}

\begin{table}[h]
\centering
\caption{Robustness check coefficients}
\label{tab:robust}
\small
\begin{tabular}{lcccc}
\hline
\textbf{Variable} & \textbf{Original $\beta$} & \textbf{Robust $\beta$} & \textbf{Change (\%)} & \textbf{p-value} \\
\hline
\multicolumn{5}{c}{\textit{Stage 3}} \\
Avg sleep hours & -0.06948 & -0.09062 & 30.43 & 5.113e-13 \\
Insomnia & 0.74061 & 0.91693 & 23.81 & 2.060e-94 \\
Sleep apnea & 0.15766 & 0.11482 & -27.17 & 3.050e-04 \\
\hline
\multicolumn{5}{c}{\textit{Stage 5}} \\
Weekday sleep & -0.01029 & -0.02794 & 171.617 & 5.517e-03 \\
Weekend sleep & -0.02802 & -0.02930 & 4.587 & 1.689e-03 \\
\hline
\end{tabular}
\end{table}

\clearpage

\begin{thebibliography}{99}

% =========================================================
% A. Background: Sleep and Depression Literature
% =========================================================

\bibitem{baglioni2011}
Baglioni, C., Battagliese, G., Feige, B., Spiegelhalder, K., Nissen, C., Voderholzer, U., Lombardo, C., \& Riemann, D. (2011).
Insomnia as a predictor of depression: A meta-analytic evaluation of longitudinal epidemiological studies.
\textit{Journal of Affective Disorders}, 135(1--3), 10--19.
\href{https://doi.org/10.1016/j.jad.2011.01.011}{doi:10.1016/j.jad.2011.01.011}. 

\bibitem{fang2019}
Fang, H., Tu, S., Sheng, J., \& Shao, A. (2019).
Depression in sleep disturbance: A review on a bidirectional relationship, mechanisms and treatment.
\textit{Journal of Cellular and Molecular Medicine}, 23(4), 2324--2332.
\href{https://doi.org/10.1111/jcmm.14170}{doi:10.1111/jcmm.14170}.

\bibitem{dong2022}
Dong, L., Xie, Y., \& Zou, X. (2022).
Association between sleep duration and depression in US adults: A cross-sectional study.
\textit{Journal of Affective Disorders}, 296, 183--188.
\href{https://doi.org/10.1016/j.jad.2021.09.075}{doi:10.1016/j.jad.2021.09.075}.

\bibitem{zhai2015}
Zhai, L., Zhang, H., \& Zhang, D. (2015).
Sleep duration and depression among adults: A meta-analysis of prospective studies.
\textit{Depression and Anxiety}, 32(9), 664--670.
\href{https://doi.org/10.1002/da.22386}{doi:10.1002/da.22386}.

\bibitem{wheaton2012}
Wheaton, A. G., Perry, G. S., Chapman, D. P., \& Croft, J. B. (2012).
Sleep disordered breathing and depression among U.S. adults: NHANES, 2005--2008.
\textit{Sleep}, 35(4), 461--467.
\href{https://doi.org/10.5665/sleep.1724}{doi:10.5665/sleep.1724}.

\bibitem{edwards2020}
Edwards, C., Almeida, O. P., \& Ford, A. H. (2020).
Obstructive sleep apnea and depression: A systematic review and meta-analysis.
\textit{Maturitas}, 142, 45--54.
\href{https://doi.org/10.1016/j.maturitas.2020.06.002}
{doi:10.1016/j.maturitas.2020.06.002}.


% =========================================================
% B. Data + Measurement (NHANES + PHQ-9 + Sleep Questionnaire Docs)
% =========================================================

\bibitem{kroenke2001}
Kroenke, K., Spitzer, R. L., \& Williams, J. B. W. (2001).
The PHQ-9: Validity of a brief depression severity measure.
\textit{Journal of General Internal Medicine}, 16(9), 606--613.
\href{https://doi.org/10.1046/j.1525-1497.2001.016009606.x}{doi:10.1046/j.1525-1497.2001.016009606.x}.

\bibitem{stierman2021}
Stierman, B., Afful, J., Carroll, M. D., et al. (2021).
NHANES 2017--March 2020 prepandemic data files: Development of files and prevalence estimates for selected health outcomes.
\textit{National Health Statistics Reports}, No. 158. National Center for Health Statistics.
\href{https://doi.org/10.15620/cdc:106273}{doi:10.15620/cdc:106273}.

% =========================================================
% C. Methods: Negative Binomial / Count-Data Regression
% =========================================================

\bibitem{hilbe}
Hilbe, J. M. (2011).
\textit{Negative Binomial Regression} (2nd ed.).
Cambridge University Press. % :contentReference[oaicite:11]{index=11}

\bibitem{camerontrivedi}
Cameron, A. C., \& Trivedi, P. K. (2013).
\textit{Regression Analysis of Count Data} (2nd ed.).
Cambridge University Press. % :contentReference[oaicite:12]{index=12}

\end{thebibliography}


\end{document}
